\chapter{马克思恩格斯著作}
\Large\textbf{Works of Karl Marx and Fredrich Engels}\normalsize

\section{共产党宣言}

\par 《共产党宣言》的\textbf{基本思想}(\emph{唯物史观}):
\begin{enumerate}
    \item 每一历史时代的经济生产及其必然产生的社会结构,是该时代政治和精神历史的基础。
    \item 原始土地公有制解体以来的全部历史都是阶级斗争(被剥削阶级与剥削阶级,被统治阶级与统治阶级的斗争)的历史。
    \item 现代大工业产生的无产阶级,只有使整个社会永远摆脱剥削、压迫和阶级斗争,才能使自己从资产阶级的剥削和压迫之下解放出来。
\end{enumerate}

\par 随着需求的增加、市场的扩大,工业生产方式由封建行会依次转变为工场手工业、现代大工业。阶级对立简单化,社会分裂为直接对立的资产阶级和无产阶级。

\par 《宣言》的\textbf{目的}是宣告现代资产阶级所有制必然灭亡。其原因有二:(1)资产阶级所有制下自由竞争导致生产过剩,产生周期性的商业危机,无产者失业,资产者破产;(2)劳动的价格与其生产费用一致,从而工人失去财产、变为赤贫者,仅足以维持生活、延续后代,造成革命的因素。\emph{从供需角度看,劳动力供过于求使上述情况发生,而这是生产力发展的自然后果。}

\par \textbf{共产党的理论}:代表整个无产阶级的利益、整个共产主义运动的利益。最近目的是推翻资产阶级统治,建立无产阶级政权。理论核心是消灭私有制(并非剥夺占有社会产品的权力,而是剥夺利用这种占有奴役他人的权力)。在共产主义社会,一切生产部门由整个社会经营,工业生产由整个社会按照确定的计划和所有人的需要来领导;“每个人的自由发展是一切人自由发展的条件”\footnote{恩格斯的《共产主义原理》将全面发展解释为工种轮换,摆脱分工带来的片面性。}。

\par \textbf{历史背景}:《宣言》发表于1848年,是马克思和恩格斯应邀为共产主义者同盟撰写的党纲。后者前身是成立于1836年的正义者同盟,1847年在马克思、恩格斯指导下改组为第一个以科学社会主义为指导的无产阶级政党,1852年科隆共产党人案件后解散。其为1864年成立的国际工人协会(第一国际)培养了人才。

\par 在1847年,“社会主义是资产阶级的运动,共产主义是工人阶级的运动。”前者寄希望于统治阶级,如欧文、傅里叶的空想社会主义;后者则不局限于政治变革,主张根本改造社会,如卡贝、魏特林。